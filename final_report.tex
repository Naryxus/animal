\documentclass[solution]{tudreport}
\usepackage[utf8]{inputenc}
\usepackage[german]{babel}
\usepackage{amsmath}
\usepackage{tabularx}
\usepackage{booktabs}
\usepackage{float}
\usepackage{enumerate}
\usepackage{rotating}
\usepackage{listings}
\title{Visualisierung von Algorithmen und Datenstrukturen}
\subtitle{Abschlussbericht}
\subsubtitle{Gruppe 33: Benedikt Lins (1799381) und Stefan Thaut (1800351)\\
			Fachbereich 20 - Informatik\\
			\today}
\begin{document}
	\lstset{language=Java}
	\maketitle
	
	\chapter{Lernerfolge}
		In diesem Praktikum sollten verschiedene Datenstrukturen und Algorithmen visualisiert werden. Dazu sollte das in Java geschriebene Framework \textit{Animal} genutzt werden. Dieses Framework bietet verschiedene grafische Primitive an, um sowohl den Zustand einer Datenstruktur oder Algorithmus als auch die verschiedenen Vorgänge während einer Operation der Datenstruktur oder der Algorithmen visualisieren zu können.\\
		Neben einer Auffrischung in objekt-orientierter Programmierung sowie der Funktionsweise verschiedener Datenstrukturen und Algorithmen - in unserem Fall die Dial-Implementierung der beschränkten monotonen Prioritätswarteschlange und dem Edmonds-Karp Algorithmus - haben wir den Umgang mit Animal gelernt und damit auch, was in einem Algorithmus alles visualisiert werden muss. Selbst kleinere Veränderungen des Zustands sollten hervorgehoben werden, um das Verständnis des Betrachters zu fördern.
		
	\chapter{Bewertung von Animal}
		Animal bietet eine ausreichende Menge an Visualisierungsmöglichkeiten an für die grundlegende Darstellung von Datenstrukturen und Algorithmen. Verbesserungsmöglichkeiten sehen wir nur punktuell.
		\section{Dokumentation}
			Bei einigen Klassen und Methoden, die wir genutzt haben, fehlen teilweise Informationen oder die Dokumentation gänzlich. Ein gutes Beispiel dafür sind die Parameter \textit{direction} und \textit{moveType} der Methoden \textit{moveVia} und \textit{moveTo}. Hier ist zwar beschrieben, dass die beiden Parameter die Richtung, in der das Primitiv bewegt werden soll, bzw. den Typ der Bewegung angeben sollen. Leider sind aber nicht die möglichen Werte angegeben, die der jeweilige String annehmen kann. Dies verzögert die Entwicklung, da zunächst entweder per Trial and Error getestet werden muss, was man einsetzen kann, oder sich andere Generatoren anschauen muss, wie der jeweilige Code aussehen muss.
		
		\section{Label für die Inhaltsangabe}
			Während der Überarbeitung der Algorithmen, um Einträge in der Inhaltsangabe einzufügen, sind wir auf die Kuriosität gestoßen, dass man den Titel für den Schritt, den man in der Inhaltsangabe aufführen möchte, in der \lstinline!nextStep()!-Methode \textbf{nach} dem jeweiligen Schritt angeben muss:
			\begin{lstlisting}[frame=single, caption = Code-Beispiel für die Angabe des Titels nach dem Code]
lang.nextStep();

//Code for important Step

lang.nextStep("Title for important Step");
			\end{lstlisting}
			Dies ist für den Programmierer eher unintuitiv, da man davon ausgeht, dass das was nach dem Aufruf folgt, unter dem entsprechenden Titel zu finden ist. Wir vermuten, dass dies daran liegt, dass man für den ersten Schritt (i.d.R. die Titelfolie) die \lstinline!nextStep()!-Methode nicht aufruft und so keinen Eintrag in der Inhaltsangabe erzeugen könnte.
		
		\section{Erweiterungsmöglichkeit der Primitive}
		
		\section{Übersicht der Generatoren}
			Bei der Recherche bereits vorhandener Generatoren ist uns aufgefallen, dass die Suche nach bestimmten Generatoren in der Übersicht in Animal schwierig ist. Auch wenn für alle Generatoren eine Kurzbeschreibung in der Übersicht angezeigt wird, ist diese manchmal nur wenig aussagend oder unverständlich formuliert, sodass man den Generator erst einmal starten muss, um zu erkennen, was der Algorithmus tatsächlich leisten soll. Auch die Namen sind manchmal unglücklich gewählt, sodass man bei manchen Algorithmen fälschlicherweise einen anderen Algorithmus erwartet als den, den man dann letztendlich startet.\\
			Darüber hinaus ist es bei manchen Algorithmen schwer, diese in eine passende Kategorie einzuordnen. Der Algorithmus von Edmonds und Karp beispielsweise kann sowohl in die Kategorie der Graphenalgorithmen als auch in Algorithmen für Netzwerke eingeordnet werden. Hier könnte eine konsistentere Kategorisierung vorgenommen werden.
	
	\chapter{Bewertung der Organisation}
		Die Organisation des Praktikums kann anstandslos als einwandfrei bezeichnet werden. Die Aufgabenstellungen sind offen gestellt, sodass man sich die Bearbeitungszeit des Praktikums vollkommen frei einteilen kann. So ist es sowohl möglich, das Praktikum in den ersten Wochen des Semesters zu absolvieren als auch die Prüfungen des Semesters abzuwarten, um sich besser auf diese vorbereiten zu können und das Praktikum in deren Anschluss zu bearbeiten.\\
		Eine weitere Unterstützung der Studenten bietet die Möglichkeit, zwischen Online- und Präsenztestaten zu wählen. So werden Studenten, die möglicherweise nicht in der näheren Umgebung der Universität wohnen oder nicht sehr flexibel in der Terminauswahl sind, nicht gezwungen, in der Universität erscheinen zu müssen. Auf der anderen Seite wird Studenten, die Hilfe vor Ort benötigen, durch die Testate die Möglichkeit gegeben, ein Gespräch zu führen.\\\\
		Die Kommunikation über Moodle erfolgte reibungslos. Man erhielt üblicherweise sehr schnell eine, insbesondere hilfreiche, Antwort oder wusste wahlweise, dass der betreuende Tutor für einen gewissen Zeitraum abwesend ist. Der Tutor war darüber hinaus äußerst flexibel, was Termine anbelangt, sodass er selbst weitere Termine ermöglichte, obwohl bereits alle Termine, die in Moodle aufgeführt waren, belegt waren.\\\\
		Die Praktikumsblätter waren für den Einstieg hilfreich. Für die spätere Entwicklung der eigenen Generatoren wäre aber vielleicht eine Art Wiki hilfreicher, in dem man dann speziell nach den Themen suchen kann, die man gerade benötigt. Hier war die Suche in den Praktikumsblättern eher mühsam. Manche Beispiele der Praktikumsblätter wären noch hilfreicher gewesen, wenn sie konkreter auf die Animal-API eingegangen wären. Ein gutes Beispiel hierfür ist die Funktionsweise des Translators, der in den Blättern anhand von GUI-Komponenten erklärt wird, sodass wir uns den Code anderer Generatoren anschauen mussten, um die genaue Anwendung des Translators in Bezug auf Animal verstehen zu können.\\\\
		Abschließend kann man zusammenfassen, dass das Praktikum viel Spaß gemacht hat, da es eine breite Sammlung von verschiedenen Themen des Informatikstudiums abdeckt, in die man sich wieder einmal hineintasten konnte. Für die Zukunft kann man sich wünschen, dass Animal und die implementierten Algorithmen auch von anderen Fachgruppen in deren Lehre eingesetzt werden und Animal dadurch noch populärer wird.
\end{document}